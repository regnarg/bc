%%% The main file. It contains definitions of basic parameters and includes all other parts.

%% Settings for single-side (simplex) printing
% Margins: left 40mm, right 25mm, top and bottom 25mm
% (but beware, LaTeX adds 1in implicitly)
%\documentclass[12pt,a4paper]{report}
%\setlength\textwidth{145mm}
%\setlength\textheight{247mm}
%\setlength\oddsidemargin{15mm}
%\setlength\evensidemargin{15mm}
%\setlength\topmargin{0mm}
%\setlength\headsep{0mm}
%\setlength\headheight{0mm}
%% \openright makes the following text appear on a right-hand page
%\let\openright=\clearpage

%% Settings for two-sided (duplex) printing
 \documentclass[12pt,a4paper,twoside,openright]{report}
 \setlength\textwidth{145mm}
 \setlength\textheight{247mm}
 \setlength\oddsidemargin{14.2mm}
 \setlength\evensidemargin{0mm}
 \setlength\topmargin{0mm}
 \setlength\headsep{0mm}
 \setlength\headheight{0mm}
 \let\openright=\cleardoublepage

%% Generate PDF/A-2u
\usepackage[a-2u]{pdfx}

%% Character encoding: usually latin2, cp1250 or utf8:
\usepackage[utf8]{inputenc}

\usepackage[czech,english]{babel}

%% Prefer Latin Modern fonts
\usepackage{lmodern}

%% Further useful packages (included in most LaTeX distributions)
\usepackage{amsmath}        % extensions for typesetting of math
\usepackage{amsfonts}       % math fonts
\usepackage{amsthm}         % theorems, definitions, etc.
\usepackage{bbding}         % various symbols (squares, asterisks, scissors, ...)
\usepackage{bm}             % boldface symbols (\bm)
\usepackage{graphicx}       % embedding of pictures
\usepackage{fancyvrb}       % improved verbatim environment
%\usepackage{natbib}         % citation style AUTHOR (YEAR), or AUTHOR [NUMBER]
\usepackage[nottoc]{tocbibind} % makes sure that bibliography and the lists
			    % of figures/tables are included in the table
			    % of contents
\usepackage{dcolumn}        % improved alignment of table columns
\usepackage{booktabs}       % improved horizontal lines in tables
\usepackage{paralist}       % improved enumerate and itemize
\usepackage[usenames]{xcolor}  % typesetting in color
\usepackage[
   backend=biber        % if we want unicode
  ,style=iso-numeric
  ,autolang=other       % to support multiple languages in bibliography
  ,sortlocale=en_US     % locale of main language, for sorting
  ,bibencoding=UTF8     % this is necessary only if bibliography file is in different encoding than main document
]{biblatex}
\addbibresource{bibliography.bib}


%%% Basic information on the thesis

% Thesis title in English (exactly as in the formal assignment)
\def\ThesisTitle{A decentralized file synchronization tool}

% Author of the thesis
\def\ThesisAuthor{Filip Štědronský}

% Year when the thesis is submitted
\def\YearSubmitted{2017}

% Name of the department or institute, where the work was officially assigned
% (according to the Organizational Structure of MFF UK in English,
% or a full name of a department outside MFF)
\def\Department{Department of Applied Mathematics}

% Is it a department (katedra), or an institute (ústav)?
\def\DeptType{Department}

% Thesis supervisor: name, surname and titles
\def\Supervisor{Mgr. Martin Mareš, Ph.D.}

% Supervisor's department (again according to Organizational structure of MFF)
\def\SupervisorsDepartment{Department of Applied Mathematics}

% Study programme and specialization
\def\StudyProgramme{Computer Science}
\def\StudyBranch{General Computer Science}

% An optional dedication: you can thank whomever you wish (your supervisor,
% consultant, a person who lent the software, etc.)
\def\Dedication{%

To Medvěd, my supervisor, a great friend and one of the best teachers I~know.
Someone who has way more answers than one person should. Whenever I~encounter
a problem from pretty much any field, the first subconscious impulse usually is:
``Let's ask Medvěd, he will figure something out.'' But I learn to resist this
impulse and instead try to acquire some of the tricks of his trade -- relentless
curiosity being one of the most important. Do not stop with a half-baked
kinda-sorta answer. Think things through. Experiment. Poke. Change assumptions.
Ask nagging questions.

To Karry, one of the few close friends I have ever had. An endless source of amazement,
sometimes rumoured to have supernatural powers. She actually
managed to get two master's degress almost faster than I will (hopefully) get
my bachelor's! An inspiration to dare do (not try to do, simply do) more seemingly impossible things.
You have made my life better in more ways than you can imagine.

To my dad, who is always supportive, even though he often thinks I'm crazy.

To all the random happenstances of evolution that gave us the ability to write
bachelor theses and do a lot of other interesting stuff.
}

% Abstract (recommended length around 80-200 words; this is not a copy of your thesis assignment!)
\def\Abstract{%
  We explore the problem of file synchronization, with the goal of improving
  on the efficiency, scalability, robustness, flexibility and security of
  current file synchronization tools. We solve several important subproblems
  that may help this, especially in the areas of filesystem change detection
  (both online and offline) and peer-to-peer synchronization of file metadata.
  We show techniques to make scanning a file system for changes faster and
  more reliable.
  We extend the Linux kernel's `fanotify` filesystem change notification API
  to report more events, especially renames. We present several original solutions
  to the set reconciliation problem and its variants and apply them to metadata
  synchronization.
}

% 3 to 5 keywords (recommended), each enclosed in curly braces
\def\Keywords{%
  {file synchronization} {set reconciliation} {fanotify}
}

%% The hyperref package for clickable links in PDF and also for storing
%% metadata to PDF (including the table of contents).
%% Most settings are pre-set by the pdfx package.
\hypersetup{unicode}
\hypersetup{breaklinks=true}

\usepackage[noend]{algpseudocode} % requires texlive-science
\usepackage[chapter]{algorithm}

% Definitions of macros (see description inside)
\include{macros}

\usepackage{longtable}

\usepackage{unixode}


% Title page and various mandatory informational pages
\begin{document}
\include{title}

%%% A page with automatically generated table of contents of the bachelor thesis

\tableofcontents

%%% Each chapter is kept in a separate file
\include{intro}
\include{changedet}
\include{mdsync}
\include{datasync}

\include{impl}
%\include{related}

\chapter*{Conclusion}
\addcontentsline{toc}{chapter}{Conclusion}

% TEMPORARY HACK: BibLaTeX refuses to work when there are no citations.
%\phantom{\citet[Věta 4.22]{matfyzak}}
\phantom{\cite{rsync}\cite{partrecon}\cite{gentili}}


%%% Bibliography
\include{bibliography}

%%% Figures used in the thesis (consider if this is needed)
%\listoffigures

%%% Tables used in the thesis (consider if this is needed)
%%% In mathematical theses, it could be better to move the list of tables to the beginning of the thesis.
%\listoftables

%%% Abbreviations used in the thesis, if any, including their explanation
%%% In mathematical theses, it could be better to move the list of abbreviations to the beginning of the thesis.
\chapwithtoc{List of Abbreviations}

\vbox{\noindent\textbf{FCV} -- file content version (sec. \ref{sec:objects}).}
\vbox{\noindent\textbf{FLV} -- FOB location version (sec. \ref{sec:objects}).}
\vbox{\noindent\textbf{FOB} -- filesystem object (sec. \ref{sec:objects}).}
\vbox{\noindent\textbf{IID} -- inode identifier (sec. \ref{sec:dirtree}).}
\vbox{\noindent\textbf{mtime} -- an inode's last modification time, as reported by the \texttt{lstat} syscall}
\vbox{\noindent\textbf{NFS} -- the Network File System \cite{nfs-rfc}}
\vbox{\noindent\textbf{OFD} -- open file description, in internal kernel structure describing an open file}
\vbox{\noindent\textbf{RTT} -- network round-trip time (i.e., what \texttt{ping} measures)}
\vbox{\noindent\textbf{syscall} -- system call, a function implemented by the kernel that can be invoked from user space}

%%% Attachments to the bachelor thesis, if any. Each attachment must be
%%% referred to at least once from the text of the thesis. Attachments
%%% are numbered.
%%%
%%% The printed version should preferably contain attachments, which can be
%%% read (additional tables and charts, supplementary text, examples of
%%% program output, etc.). The electronic version is more suited for attachments
%%% which will likely be used in an electronic form rather than read (program
%%% source code, data files, interactive charts, etc.). Electronic attachments
%%% should be uploaded to SIS and optionally also included in the thesis on a~CD/DVD.
\appendix
\chapter{Attachments}

\textbf{Attachment 1} (\texttt{filoco-0.1.tar.gz}) is a part of the electronic
version of this thesis. It contains the source code of the Filoco implementation
sketch, several experiments and proofs of concepts, and the \texttt{FANOTIFY\_MODIFY\_DIR}
kernel patches.

\openright
\end{document}
