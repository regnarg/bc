\documentclass[12pt,a4paper]{report}
\usepackage[a-2u]{pdfx}
\usepackage[utf8]{inputenc}
\usepackage[czech]{babel}
\usepackage{lmodern}

\begin{document}
V práci zkoumáme problém synchronizace souborů s cílem zlepšení efektivity,
škálovatelnosti, robustnosti, flexibility a bezpečnosti současných synchronizačních
nástrojů. Vyřešíme několik podstatných podproblémů, které tomuto mohou pomoci, zejména
v oblasti sledování změn souborového systému (online i offline) a peer-to-peer
synchronizace souborových metadat. Ukážeme techniky pro rychlejší a spolehlivější
hledání změn v souborových systémech. Rozšíříme \texttt{fanotify}, rozhraní linuxového
jádra pro oznamování změn v souborovém systému, tak, aby dokázalo oznamovat více druhů
událostí, zejména přejmenování souborů. Představíme originální řešení několika variant
problému set reconciliation (synchronizace množin) a aplikujeme je na synchronizaci
metadat.
\end{document}
